\documentclass{article}
\usepackage{polski}
\usepackage[utf8]{inputenc}
\usepackage{hyperref}
\hypersetup{
    colorlinks,
    citecolor=black,
    filecolor=black,
    linkcolor=black,
    urlcolor=black
}

\title{
    WMI Adventure - grywalizacja studiowania Informatyki \\
    \large Zarządzanie projektem i metodyka zwinna
    }
\author{Marcin Kostrzewski}
\date{22 Listopada, 2021r}

\begin{document}

\maketitle
\newpage
\tableofcontents
\newpage


\section{Wstęp}

\subsection{Streszczenie}
Zarządzanie projektami informatycznymi zaczęło mnie interesować już na 3 roku studiów, gdzie miałem przyjemność uczestniczyć w zajęciach z Inżynierii Oprogramowania prowadzonych przez
dr. Krzysztofa Krzywdzińskiego, w grupie zajęciowej u dr. Patryka Żywicy. WMI Adventure, będący pierwszym większym projektem który realizowałem drużynowo, byłem w dużym stopniu odpowiedzialny za
proces który implementowaliśmy.

Napisałem tę pracę korzystając ze zdobytej wcześniej wiedzy między innymi na zajęciach uczelnianych, z przeczytanych książek,
oraz z doświadczenia zdobytego podczas WMI Adventure. Omawiam w niej różne metodyki zwinne, porównuje je do dawnego, kaskadowego podejścia przedstawiając ich tło historyczne. Przedstawiam też cały proces w WMI Adventure, wykorzystaną metodykę i narzędzia wspomagające.

\subsection{Tematyka}
W zespołach organizacja pracy projektowej jest kluczowa dla kontrolowanego i przewidywalnego rozwoju, w tym przypadku, oprogramowania komputerowego. W systemach o mniejszym stopniu skomplikowania
i jednoosobowej sile roboczej wystarczy "tylko programowanie". W miarę poszerzania skali projektu i zwiększania ilości członków zespołu, wytwarzanie oprogramowania staje się chaotyczne i niekontrolowane. Klient, dla którego realizowany jest projekt potrzebuje wspomnianej przewidywalności. Chce chociażby wiedzieć, ile czasu potrwa realizacja jego zlecenia i przede wszystkim jaki będzie jego koszt. Zespół nie jest w stanie
rejestrować postępu prac. Każdy z członków ma inną wizję projektu i priorytetuje zupełnie inne rzeczy, niż te, które są faktycznie istotne dla klienta. Nie wiadomo, nad czym zespoł w danej chwili pracuje.
Potrzeba zatem pewnej struktury, która sprawi, że rozwój oprogramowania będzie przewidywalny - tym właśnie jest zarządzanie projektem, a metodyki to zbiór pewnych ram, które ułatwią jego rozwój.

\subsection{Cel}
WMI Adventure było piaskownicą, którą wykorzystałem do nauki zarządzania zespołem w praktyce. Analizując skutki które przynosiły moje (i innych członków zespołu) decyzje, starałem się z każdym tygodniem usprawniać
proces, wynosząc przy tym wiedzę którą chcę tutaj przekazać. W litetaturze czy materiałach dostępnych w internecie możemy znaleźć ogólne zasady zarządzania projektem, które też tutaj przytaczam, natomaist
meritum pracy znajduje się w opisaniu konkretnego wykorzystania tych zasad, które omawiam w jej drugiej części. Drużyna WMI Adventure liczyła początkowo 4 członków, końcowo trzech. Koordynacja mniejszymi zespołami wcale nie musi być prostsza niż tymi dużymi. Chcę w tej pracy opisać metodyki w oparciu o małe zespoły i dostarczyć informację na temat problemów i potencjalnych rozwiązań, które w takich drużynach
występują.

\section{Metodyki pracy i ich kontekst historyczny}

\subsection{Metodyki kaskadowe}
\subsubsection{Role}
\subsubsection{Organizacja pracy}
\subsubsection{Cykl życia projektu}

\subsection{Agile}
\subsubsection{Manifest Agile}
\subsubsection{Cykl życia projektu}

\subsection{Porównanie metodyk zwinnych i kaskadowych}

\section{Metodyki zwinne}

\subsection{Scrum}
\subsubsection{Role w zespole}
\subsubsection{Organizacja pracy}

\subsection{Kanban}
\subsubsection{Tablica Kanbanowa}

\subsection{Scrumban}

\section{Proces w WMI Adventure}

\subsection{Metodyka}
\subsection{Role w zespole}
\subsection{User Stories}

\section{Wykorzystanie GitHub Projects jako narzędzia do zarządzania procesem}

\section{Ewolucja}

\end{document}
